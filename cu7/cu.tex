% \IUref{IUAdmPS}{Administrar Planta de Selección}
% \IUref{IUModPS}{Modificar Planta de Selección}
% \IUref{IUEliPS}{Eliminar Planta de Selección}

% 


% Copie este bloque por cada caso de uso:
%-------------------------------------- COMIENZA descripción del caso de uso.

%\begin{UseCase}[archivo de imágen]{UCX}{Nombre del Caso de uso}{
	\begin{UseCase}{CU7}{Agregar Medicamento}{
		Un proveedor se encarga de suministrar de medicamentos a la farmacia. Así, el proveedor es quien incrementa el stock de uno o varios medicamentos determinados
	}
		\UCitem{Versión}{1.3}
		\UCitem{Actor}{Proveedor}
		\UCitem{Propósito}{Que el actor pueda registrar los cambios que hace en el stock de la farmacia.}
		\UCitem{Resumen}{
		Mediante el ID de un medicamento, el actor puede registrar la cantidad de unidades que est\'a suministrando a la farmacia de tal medicamento. }
		\UCitem{Entradas}{ID y cantidad de medicamento a agregar}
		\UCitem{Salidas}{Cantidad del medicamento que hay en el stock tras ejecutar el cambio}
		\UCitem{Precondiciones}{El médicamento que se desea suministrar debe estar registrado en el sistema.}
		\UCitem{Postcondiciones}{El stock del medicamento aumentará, de acuerdo a la cantidad ingresada por el actor.}
		\UCitem{Errores}{El medicamento no esta registrado en el sistema.}
		\UCitem{Autor}{Ricardo Cuéllar Sánchez}
		\UCitem{Revisor}{Gerardo Aramis Cabello Acosta}
		\UCitem{Status}{Revisado}
	\end{UseCase}

	\begin{UCtrayectoria}{Principal}
		\UCpaso[\UCactor] Introduce el ID del medicamento cuyo stock se desea aumentar, en la\IUref{IUProv}{Pantalla de Proveedor}\label{CU7IntroduceMedicamentos}.
		\UCpaso[\UCactor] Ingresa la cantidad que se desea agragar a ese medicamento. 
		\UCpaso[\UCactor] Selecciona la opción ``Agregar''. \Trayref{A}
		\UCpaso El sistema suma el stock actual m\'as la cantidad ingresada, y lo guarda como nuevo stock del medicamento ingresado.
		\UCpaso El sistema muestra el mensaje {\bf OKStock}.
	\end{UCtrayectoria}
		
		\begin{UCtrayectoriaA}{A}{El sistema no encuentra el medicamento del ID ingresado.}
			\UCpaso El sistema muestra el mensaje {\bf ERRMedInex-}
			\UCpaso[\UCactor] oprime el botón \IUbutton{Aceptar}.  
			\UCpaso[\UCactor] regresa al paso 1 de la trayectoria principal.  
		\end{UCtrayectoriaA}
		
		
	
		
%-------------------------------------- TERMINA descripción del caso de uso.