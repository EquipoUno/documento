% \IUref{IUAdmPS}{Administrar Planta de Selección}
% \IUref{IUModPS}{Modificar Planta de Selección}
% \IUref{IUEliPS}{Eliminar Planta de Selección}

% 


% Copie este bloque por cada caso de uso:
%-------------------------------------- COMIENZA descripción del caso de uso.

%\begin{UseCase}[archivo de imágen]{UCX}{Nombre del Caso de uso}{
	\begin{UseCase}{CU7}{Agregar Médicamento}{
		Un proveedor se encarga de suministrar de medicamentos a la farmacia. Así mismo el proveedor podrá registrar los cambios realizados en el stock del sistema.
	}
		\UCitem{Versión}{1.2}
		\UCitem{Actor}{Proveedor}
		\UCitem{Propósito}{Que el proveedor pueda suministrar la farmacia y registrar los cambiar realizados.}
		\UCitem{Resumen}{
		El sistema muestra los medicamentos que se han agotado al proveedor para que este pueda suministrarlos.El proveedor guardará los cambios hechos y el sistema cambia los datos en el stock. }
		\UCitem{Entradas}{ID y cantidad de medicamento agregado}
		\UCitem{Salidas}{Stock Actualizado}
		\UCitem{Precondiciones}{El médicamento que se desea suministrar debe estar registrado en el sistema.}
		\UCitem{Postcondiciones}{El stock del medicamento aumentará, de acuerdo a la cantidad ingresada por el actor.}
		\UCitem{Errores}{El medicamento no esta registrado en el sistema.}
		\UCitem{Autor}{Ricardo Cuéllar Sánchez}
		\UCitem{Revisor}{}
		\UCitem{Status}{Espera de revisión}
	\end{UseCase}

	\begin{UCtrayectoria}{Principal}
		\UCpaso[\UCactor] Introduce el ID del medicamento cuyo stock se desea aumentar, en la\IUref{UIProv}{Pantalla de Proveedor}\label{CU7IntroduceMedicamentos}.
		\UCpaso[\UCactor]Ingresa la cantidad que se desea agragar a ese medicamento. 
		\UCpaso[\UCactor] Selecciona la opción ``Agregar''. \Trayref{A}
		\UCpaso Suma el stock actual más la cantidad ingresada, y lo guarda como nuevo stock del medicamento ingresado.
		\UCpaso Muestra el mensaje Stock Actualizado
		\UCpaso[] Termina el caso de uso.
	\end{UCtrayectoria}
		
		\begin{UCtrayectoriaA}{A}{El sistema no encuenta el medicamento del ID ingresado.}
			\UCpaso Muestra el Mensaje {\bf MSG1-}``El ID [{\em ID clave}] no se encuentra registrado en el sistema''.
			\UCpaso[\UCactor] Oprime el botón \IUbutton{Aceptar}, regresa al paso anterior [{\em CU7.1}].  
			\UCpaso[] Termina la trayectoria.
		\end{UCtrayectoriaA}
		
		
	
		
%-------------------------------------- TERMINA descripción del caso de uso.