% \IUref{IUAdmPS}{Administrar Planta de Selección}
% \IUref{IUModPS}{Modificar Planta de Selección}
% \IUref{IUEliPS}{Eliminar Planta de Selección}

% 


% Copie este bloque por cada caso de uso:
%-------------------------------------- COMIENZA descripción del caso de uso.

%\begin{UseCase}[archivo de imágen]{UCX}{Nombre del Caso de uso}{
	\begin{UseCase}{CU5}{Pago de transacci\'on}{
		El cajero podr\'a finalizar transacciones pendientes (consulta o pago por medicamentos) por medio del folio de la misma.
En el caso de los medicamentos, el cajero sellar\'a el recibo otorgado por el cliente y se lo devoler\'a.
	}
		\UCitem{Versión}{1.1}
		\UCitem{Actor}{Cajero}
		\UCitem{Propósito}{Que el cajero administre las entradas y salidas de dinero con respecto a los pagos realizados en el d\'ia de citas o farmacia}
		\UCitem{Resumen}{
		El cajero tendr\'a la responsabilidad de verificar entradas y salidas de dinero, regstrando pagos para citas o para ventas de medicamentos, recibir\'a las ordenes de pago (tanto de citas como de farmacia), recibir\'a efectivo y devolver\'a a paciente el comprobante de pago}
		\UCitem{Entradas}{ID transaccion, monto recibo.}
		\UCitem{Salidas}{Cantidad de dinero a devoler. ERRCambioInsuf. OKDatosMedicamento, ERRCampoVacio.}
		\UCitem{Precondiciones}{Tener una transacci\'on pendiente en el sistema. }
		\UCitem{Postcondiciones}{La transacci\'on pasar\'a a un status "Pagado".}
		\UCitem{Errores}{No hay internet. El servidor está caído.}
		\UCitem{Autor}{Daniel Alejandro Barros Alvarado}
		\UCitem{Revisor}{Diana Guadalupe Maldonado Ledo}
		\UCitem{Status}{Terminado}
	\end{UseCase}

	\begin{UCtrayectoria}{Principal}
		\UCpaso[\UCactor] introdue el folio de la receta en la \IUref{IUCaja}{Pantalla de Caja}.
		\UCpaso[\UCactor] recibe del cliente, un monto de dinero.
		\UCpaso[\UCactor]introduce la cantidad recibida en el campo "Monto" de la \IUref{IUCaja}
		\UCpaso El sistema calcular\'a el cambio (en caso de ser necesario)\Trayref{A} 
		\label{IUConsulta} .
		\UCpaso[\UCactor] El actor entrega cambio y comprobante de pago.
		\UCpaso [\UCactor] selecciona la opción "`Aceptar"'.  
		\UCpaso El sistema actualiza el stock en caja y farmacia. 
		\UCpaso[] Termina el caso de uso.
	\end{UCtrayectoria}
		
		\begin{UCtrayectoriaA}{A}{El cambio que se requiere entregar no es suficiente en Caja}
			\UCpaso Muestra el mensaje {\bf ERRCambioInsufucuente}.
			\UCpaso[\UCactor] oprime el botón \IUbutton{Aceptar}
			\UCpaso[\UCactor] oprime el botón \IUbutton{Cancelar Pago}
			\UCpaso[\UCactor] regresa al paso 1 de la trayectoria principal.  
			\UCpaso[] Termina la trayectoria.
		\end{UCtrayectoriaA}
		
		
%-------------------------------------- TERMINA descripción del caso de uso.