\begin{UseCase}{CU8}{Agendar cita}{
}
\UCitem{Versión}{1.1}
\UCitem{Actor}{Cliente}
\UCitem{Propósito}{El cliente puede agendar una cita, seleccionando el día y la hora conveniente.}
\UCitem{Resumen}{
El sistema cuenta con dos combobox en los cuales, se muestra el desglose del calendario para la selección del día y el segundo la hora (en intervalos de 30 minutos) en un horario de 7:00 a 17:30 hrs. Al finalizar la selección, existe un botón para comprobar la disponibilidad de la cita dentro del sistema y que pueda ser agendada. En caso de que no esté disponible la cita en el horario seleccionado, se mostrara un mensaje en el que se pedirá elegir otro horario u otra fecha.}
\UCitem{Entradas}{Fecha y hora}
\UCitem{Salidas}{Mensaje de confirmación de cita o de cambio de cita.}
\UCitem{Precondiciones}{Estar registrado en el sistema, no contar con alguna cita previa.}
\UCitem{Postcondiciones}{Se mostrara la información en pantalla.}
\UCitem{Errores}{Ya existe una cita agendada en ese horario y fecha {\bf ERRCita}”No se puede generar la cita, ya hay una cita en ese horario en los consultorios”. {\bf ERRConLLen}\Trayref{A}”Favor de seleccionar otro horario.”}. 
\UCitem{Autor}{Barrios Alvarado Daniel Alejandro}
\UCitem{Revisor}{}
\UCitem{Status}{Espera de revisión}
\end{UseCase}

\begin{UCtrayectoria}{Principal}
\UCpaso[\UCactor] El actor selecciona la opción “Fecha” para seleccionar el día que desea la cita y la hora; después deberá presionar la opción “Revisar Cupo”.
\UCpaso[\UCactor] El sistema muestra la pantalla de Confirmacion de Cita. 
\end{UCtrayectoria}

\begin{UCtrayectoriaA}{A}{Horario y fecha ya ocupada.}
\UCpaso Muestra el Mensaje {\bf MSG1-}``Favor de seleccionar otro horario o fecha, los consultorios no están disponibles ” [{\em ConsultorioLLeno }].
\UCpaso[\UCactor] Oprime el botón \IUbutton{Escoger otro horario}, regresa al paso anterior [{\em CU8.1}]. 
\UCpaso[] Termina la trayectoria.
\end{UCtrayectoriaA}
