
%---------------------------------------------------------
\section{Reglas de Negocio}

\begin{BussinesRule}{BR129}{Determinar si un Paciente puede recibir consulta.} 
	\BRitem[Descripci�n:] El paciente requiere llegar 10 min. antes de que inicie su consulta y pagarla.
	\BRitem[Tipo:] Restricci�n de integridad y operacion.
	\BRitem[Nivel:] Obligatorio.
\end{BussinesRule}

\begin{BussinesRule}{BR130}{Determinar si un Paciente cuenta historial medico}
	\BRitem[Descripci�n:] Si es la primera vez que el paciente asiste a la clinica debe llenar su historial medico.
	\BRitem[Tipo:] Restricci�n de operaci�n.
	\BRitem[Nivel:] Obligatorio.
\end{BussinesRule}

\begin{BussinesRule}{BR143}{Validar los dotores disponibles}
	\BRitem[Descripci�n:] Una vez que el paciente cuente con los requisitos anteriores, se debra verificar si el doctor esta, en caso de que no este se le dara la opcion al paciente de ir otro dia o que lo atienda otro doctor.
	\BRitem[Tipo:] Restricci�n de operaci�n.
	\BRitem[Nivel:] Obligatorio.
\end{BussinesRule}

\begin{BussinesRule}{BR144}{Venta de medicamentos}
	\BRitem[Descripci�n:] La venta de medicamentos puede realizarse sin receta, en caso de necesitar un antibiotico sera necesario presentar la receta original.
	\BRitem[Tipo:] Restricci�n de operaci�n.
	\BRitem[Nivel:] Obligatorio.
\end{BussinesRule}

\begin{BussinesRule}{BR180}{Cobro de medicamentos}
	\BRitem[Descripci�n:] Los servicios se cobran de la siguiente forma:
		\begin{Citemize}
			\item {\em Publico en general:} Se les cobran todos los servicios al 100\% de su costo.
			%\item {\em Personas de 60 a�os o mas:} Se les otorga un 10\% de descuento en el costo de cada uno los medicamentos (antes del IVA).
			%\item {\em Estudiantes extranjeros:} Se les cobran los servicios al 200\% del costo registrado.
		\end{Citemize}
	\BRitem[Sentencia:] $\forall~e~\in~\mathbb{E}\textrm{Pacientes}~\land~\forall~s~\in \mathbb{S}\textrm{Medicamentos}~\Rightarrow$
		\begin{displaymath}
			Costo(p,m) = \left\{ \begin{array}{ll}
			s.costo & , si~e.tipo = \textrm{Publico general}\\
			%{s.costo}\over{5} & , si~e.tipo = \textrm{Personas de 60 a�os o mas}\\
			%s.costo \cdot 2 & , si~e.tipo = \textrm{Estudiante extranjero}
			\end{array} \right.
		\end{displaymath}

	\BRitem[Tipo:] C�lculo.
	\BRitem[Nivel:] Obligatorio.
\end{BussinesRule}

%\begin{BussinesRule}{BR45}{Calcular impuestos por consulta}
%	\BRitem[Descripci�n:] Los impuestos son al 16\% correspondientes al IVA.
%	\BRitem[Sentencia:] $Impuesto(e, s) = Costo(e, s)\cdot0.16$.
%	\BRitem[Tipo:] C�lculo.
%	\BRitem[Nivel:] Obligatorio.
%\end{BussinesRule}

\begin{BussinesRule}{BR100}{Recibo de paciente por cobre de medicamentos y atencion medica.}
	\BRitem[Descripci�n:] El  Recibo del paciente por atencion medica debe mostrar el total del costo con el siguiente desglose:
		\begin{displaymath}\begin{array}{lr}
			Costo: & \$ XXX.XX\\
			%Descuento~aplicado~(YY\%): & \$ XXX.XX\\
			Subtotal: & \$ XXX.XX\\
			%IVA~(16\%): & \$ XXX.XX\\\hline
			Total: & \$ XXX.XX
		\end{array}\end{displaymath}
	\BRitem[Descripci�n:] El  Recibo del paciente por cobro de medicamentos debe mostrar el total del costo con el siguiente desglose:
		\begin{displaymath}\begin{array}{lr}
			Costo: & \$ XXX.XX\\
			%Descuento~aplicado~(YY\%): & \$ XXX.XX\\
			Subtotal: & \$ XXX.XX\\
			%IVA~(16\%): & \$ XXX.XX\\\hline
			Total: & \$ XXX.XX
		\end{array}\end{displaymath}	
	\BRitem[Sentencia:] $CostoTotal = Costo(e, s) + Impuesto(e, s)$.
	\BRitem[Tipo:] Restricci�n de operaci�n/C�lculo.
	\BRitem[Nivel:] Obligatorio.
\end{BussinesRule}