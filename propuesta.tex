Introducción al capítulo

%--------------------------------------------------
\section{Objetivos}

% - - - - - - - - - - - - - - - - - - - - - - - - -
\subsection{Objetivo general}
Crear un sistema en el que puedan solucionar las problem\'aticas principales en una cl\'inica m\'edica, tanto en
la cl\'inica como en el sitio web.
% - - - - - - - - - - - - - - - - - - - - - - - - -
\subsection{Objetivos específicos}
\begin{itemize}
\item El sistema llevar\'a a cabo el proceso de citas por internet, donde el paciente podr\'a seleccionar la hora y fecha en la que solicitar\'a una cita m\'edica.
\item El sistema usar\'a un repositorio de datos para contener los expedientes m\'edicos de los pacientes que tengan consulta.
\item El sistema deber\'a tener control e informar sobre el stock del almac\'en en la farmacia.
\item El sistema informar\'a al cliente si la cita puede realizarse en la fecha y hora dados, con base a la disponibilidad de consultorios. En
caso de que est\'en agotada en un horario y fecha determinados, el paciente podr\'a volver a solicitar una cita.
\item 
\item 
\item 
\item 
\item 
\item 
\end{itemize}
\hspace{-.50cm}
\hspace{-.50cm}
%--------------------------------------------------
\section{Modelo de despliegue}


% - - - - - - - - - - - - - - - - - - - - - - - - -
\subsection{Requerimientos no funcionales}

\begin{figure}[htbp!]
		\centering		
	\end{figure}

\begin{table}[h]
\hspace{-.50cm}
\begin{tabular}{|l|l|l|}
\hline
	\multicolumn{1}{|c|}{\textbf{ID}} & \multicolumn{1}{c|}{\textbf{Nombre}} & \multicolumn{1}{c|}{\textbf{Descripci\'on}} \\ \hline
	RNF1 & Forma de almacenamiento & \begin{tabular}[c]{@{}l@{}}La informaci\'on requerida se guardar\'a en una base de datos de MySQL
																														 \end{tabular} \\ \hline
	RNF2	& Cupo de horario & \begin{tabular}[c]{@{}l@{}}Mediante consultas de SQL, la plataforma web determinar\'a si es posible realizar\\una cita en el horario seleccionado
																											 \end{tabular} \\ \hline
	RNF3 & Tecnolog\'ias web & \begin{tabular}[c]{@{}l@{}}Para la plataforma web se usar\'an las tecnolog\'ias HTML, CSS y PHP
																													 \end{tabular} \\ \hline
	RNF4 & \begin{tabular}[c]{@{}l@{}}Selecci\'on de horario \end{tabular} & 
	\begin{tabular}[c]{@{}l@{}}El sistema le mostrar\'a al usuario los horarios disponibles \\ en intervalos de media hora. \end{tabular} \\ \hline
																															
\end{tabular}
\caption{Requerimientos no funcionales}
\end{table}


% - - - - - - - - - - - - - - - - - - - - - - - - -
\subsection{Modelo de despliegue del sistema}


	\begin{figure}[htbp!]
		\centering
			\includegraphics[width=1.15\textwidth]{images/arquitectura1.png}
		\caption{Diagrama de arquitectura.}
	\end{figure}



% - - - - - - - - - - - - - - - - - - - - - - - - -
\subsection{Especificación de Plataforma}

\bfseries Servidor (Hardware): \mdseries
\begin{itemize}
\item Procesador AMD A10-4600M APU
\item 6 GB RAM DDR3
\item Motherboard InsydeH2O CCB.03.72.306.00
\end{itemize}

\bfseries Servidor (Software): \mdseries
\begin{itemize}
\item Windows 8 x64 (6.2, compilaci\'on 9200)
\item Apache 2.4.20
\item MySQL 5.7.12
\end{itemize}

\bfseries Cliente-clinica (Hardware): \mdseries
\begin{itemize}
\item Procesador AMD A10-4600M APU
\item 6 GB RAM DDR3
\item Motherboard InsydeH2O CCB.03.72.306.00
\end{itemize}

\bfseries Cliente-clinica (Software): \mdseries
\begin{itemize}
\item Windows 8 x64 (6.2, compilaci\'on 9200)
\item MySQL 5.7.12
\item Java Version 8 Update 91
\end{itemize}

