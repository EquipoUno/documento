% \IUref{IUAdmPS}{Administrar Planta de Selección}
% \IUref{IUModPS}{Modificar Planta de Selección}
% \IUref{IUEliPS}{Eliminar Planta de Selección}

% 


% Copie este bloque por cada caso de uso:
%-------------------------------------- COMIENZA descripción del caso de uso.

%\begin{UseCase}[archivo de imágen]{UCX}{Nombre del Caso de uso}{
	\begin{UseCase}{CU1.1}{Iniciar sesión como dependiente}{
		Ayudar a que los Estudiantes que están por terminar la carrera se puedan inscribir en un Seminario de titulación.
	}
		\UCitem{Versión}{1.2}
		\UCitem{Actor}{Dependiente}
		\UCitem{Propósito}{Que el dependiente pueda acceder a las funciones a las que tiene acceso.}
		\UCitem{Resumen}{
		El sistema muestra una pantalla de registro, donde el dependiente ingresa su informaci\'on de acceso, y si el sistema lo verifica correctamente, le dar\'a acceso como "`Dependiente"'.}
		\UCitem{Entradas}{Nombre de usuario y contraseña.}
		\UCitem{Salidas}{ --- }
		\UCitem{Precondiciones}{Tener una cuenta de tipo "Dependiente" registrada en el sistema.}
		\UCitem{Postcondiciones}{El actor tendrá acceso a las opciones que un dependiente de la farmacia tiene.}
		\UCitem{Autor}{Gerardo Aramis Cabello Acosta.}
	\end{UseCase}

	\begin{UCtrayectoria}{Principal}
		\UCpaso[\UCactor] Introduce su Nombre de usuario y Contraseña en el sistema vía la  \IUref{UILogin}{Pantalla de Control de Acceso}\label{CU1_1Login}.
		\UCpaso El sistema verifica que exista cuenta de tipo "Dependiente" con el nombre de usuario ingresado.
		\UCpaso El sistema verifica que la contraseña ingresada coincida con la de la cuenta encontrada \Trayref{A}.
		\UCpaso Se despliega en pantalla la \IUref{IUCompra}{Pantalla de Compra}.	
	\end{UCtrayectoria}
		
		\begin{UCtrayectoriaA}{A}{El sistema no encuentra coincidencias de usuario y password}
			\UCpaso Muestra el Mensaje {\bf ERRDatosIncorrectos-}``No existe ninguna cuenta con los datos ingresados. Favor de revisar la informaci\'on introducida''.
			\UCpaso[\UCactor] Oprime el botón \IUbutton{OK}.
			\UCpaso[] Termina el caso de uso.
		\end{UCtrayectoriaA}
		
		
%-------------------------------------- TERMINA descripción del caso de uso.