% \IUref{IUAdmPS}{Administrar Planta de Selección}
% \IUref{IUModPS}{Modificar Planta de Selección}
% \IUref{IUEliPS}{Eliminar Planta de Selección}

% 


% Copie este bloque por cada caso de uso:
%-------------------------------------- COMIENZA descripción del caso de uso.

%\begin{UseCase}[archivo de imágen]{UCX}{Nombre del Caso de uso}{
	\begin{UseCase}{CU1.1}{Iniciar sesión como dependiente}{
		El dependiente realiza varias funciones, tales como buscar informaci\'on de medicamentos, consultar el stock actual, emitir recibos de pago para medicamentos y buscar medicamentos por principio activo. Antes de poder acceder a \'estas acciones, tiene qu\'e acceder al sistema como "`Dependiente"'.
	}
		\UCitem{Versión}{1.2}
		\UCitem{Actor}{Dependiente}
		\UCitem{Propósito}{Que el actor pueda acceder a las funciones a las que tiene acceso un dependiente de farmacia.}
		\UCitem{Resumen}{
		El sistema muestra una pantalla de registro, donde el actor ingresa su informaci\'on de acceso, y si el sistema lo verifica correctamente como dependiente, le dar\'a acceso como "`Dependiente"'.}
		\UCitem{Entradas}{Nombre de usuario y contraseña.}
		\UCitem{Salidas}{ --- }
		\UCitem{Precondiciones}{Tener una cuenta de tipo "`Dependiente"' registrada en el sistema.}
		\UCitem{Postcondiciones}{El actor tendr\'a acceso a las opciones que un dependiente de la farmacia tiene.}
		\UCitem{Errores}{Si no hay niinguna cuenta con los datos ingresados, muestra el Mensaje {\bf ERRDatosIncorrectos-}``No existe ninguna cuenta con los datos ingresados. Favor de revisar la informaci\'on introducida''.}
		\UCitem{Autor}{Gerardo Aramis Cabello Acosta.}
		\UCitem{Revisor}{ Ricardo Cu\'ellar S\'anchez }
		\UCitem{Status}{Revisado}
	\end{UseCase}

	\begin{UCtrayectoria}{Principal}
		\UCpaso[\UCactor] Introduce su Nombre de usuario y Contraseña en el sistema vía la  \IUref{IULogin}{Pantalla de Control de Acceso}\label{CU1_1Login}.
		\UCpaso El sistema verifica que exista cuenta de tipo "`Dependiente"' con el nombre de usuario ingresado.\Trayref{A}
		\UCpaso El sistema verifica que la contraseña ingresada coincida con la de la cuenta encontrada.\Trayref{B}
		\UCpaso Se despliega en pantalla la \IUref{IUCompra}{Pantalla de Compra}.	
	\end{UCtrayectoria}
	
	\begin{UCtrayectoriaA}{A}{El usuario no existe en la Base de Datos}
			\UCpaso El sistema muestra mensaje de error {\bf ERRusuarioInexistente-}."El usuario no existe en la Base de Datos".
			\UCpaso El sistema muestra mensaje con datos de contacto del directivo.
			\UCpaso[\UCactor]Oprime el botón \IUbutton{Aceptar}.
			\UCpaso El sistema limpia todos los datos. 
		\end{UCtrayectoriaA}
	
	\begin{UCtrayectoriaA}{B}{Los datos que ingresa el usuario son incorrectos}
			\UCpaso El sistema muestra mensaje de error {\bf ERRDatosIncorrectos-}."Los datos ingresados son incorrectos".
			\UCpaso[\UCactor]Oprime el botón \IUbutton{Aceptar}.
			\UCpaso El sistema limpia todos los datos. 
		\end{UCtrayectoriaA}
		
		
%-------------------------------------- TERMINA descripción del caso de uso.