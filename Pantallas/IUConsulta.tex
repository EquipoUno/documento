\subsection{IUConsulta Pantalla de consulta}

\subsubsection{Objetivo}
	Permitir al dependiente consultar el stock de un medicamento espec\'ifico

\subsubsection{Diseño}
	Esta pantalla aparece despu\'es de seleccionar la opci\'on "`Consultar stock"' de la \IUref{IUCompra}{Pantalla de Compra}\label{IUCompra}. Aqu\'i el dependiente puede buscar un medicamento espec\'ifico por su ID el cual puede ser conocido en la pantalla de b\'usqueda.

\IUfig[.7]{gui/IUConsulta}{IUConsulta}{Pantalla de Consulta.}

\subsubsection{Salidas}

	Cantidad del medicamento que hay en stock.

\subsubsection{Entradas}
ID de medicamento.

\subsubsection{Comandos}
\begin{itemize}
		\item \IUbutton{Consultar}: El sistema buscar\'a el registro del almac\'en y mostrar\'a en pantalla la cantidad de unidades que hay del medicamento encontrado.
				\item \IUbutton{Men\'u \textgreater Comprar}: Muestra la \IUref{IUComprar}{Pantalla de Compra}\label{IUCompra}.
		\item \IUbutton{Men\'u \textgreater Consultar}: Reinicia la \IUref{IUConsulta}{Pantalla de Consulta}\label{IUConsulta}.
		\item \IUbutton{Men\'u \textgreater Salir}: Cierra la aplicaci\'on.
\end{itemize}

\subsubsection{Mensajes}
	\begin{Citemize}
		\item {\bf WARNVac\'io} - "`Ning\'un campo debe estar vac\'io. Favor de revisar la informaci\'on ingresada"'.
		\item {\bf ERRMedInex} - "`No existe medicamento con el ID ingresado. Favor de revisar sus datos."'.
	\end{Citemize}