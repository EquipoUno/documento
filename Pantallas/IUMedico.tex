\subsection{IUMedico Pantalla Medico}

\subsubsection{Objetivo}
	Permitir al M\'edico acceder a las opciones que tiene permitas dentro del men\'u.

\subsubsection{Diseño}
	Esta pantalla aparece despu\'es de iniciar sesi\'on como "`Medico"'. Aqu\'i el m\'edico podr\'a seleccionar generar reeta, crear, modificar o generar un expediente medico. Una vez seleccionada la opciones, el m\'edico podr\'a ver la pantalla desplegada correspondiente a cada opcion, y de esta manera realizar la actividad correspondiente sin algun inconveniente.

\IUfig[.5]{gui/IUMedico}{IUMedico}{Pantalla de Medico.}

\subsubsection{Salidas}

	Ninguna.

\subsubsection{Entradas}
Ninguna.

\subsubsection{Comandos}
\begin{itemize}
		\item \IUbutton{Generar Receta}: El m\'edico crea una receta medica, con base a un diagnostico ya existente para despu\'es entregar a cliente. Despu\'es muestra la \IUref{IUConfirmar}{Pantalla de Confirmaci\'on de Receta}\label{IUConfirmar}.
		\item \IUbutton{Consulta Expediente M\'edico}: Accesa a la opcion para poder hacer alguna consulta a un expediente m\'edico.
		\item \IUbutton{Generar Expediente M\'edico}: Genera desde el inicio un expediente m\'edico para alg\'un paciente nuevo.
		\item \IUbutton{Modificar Expediente M\'edico}: El m\'edico podr\'a modificar algun campo del expediente m\'edico para fines de tratamientos a inter\'ees del paciente.
		\item \IUbutton{Cerrar sesi\'on}: Cierra la sesi\'on del m\'edico.
\end{itemize}

\subsubsection{Mensajes}
	\begin{Citemize}
		\item {\bf } - "`"'.
		\item {\bf } - "`"'.
		\item {\bf } - "`"'.
		\item {\bf } - "`"'.
	\end{Citemize}