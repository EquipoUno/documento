\subsection{IUModificaExp Pantalla de Modificar Expediente}

\subsubsection{Objetivo}
	Permitir al M\'edico modificar un expediente para fines medicos y de tratamientos del paciente.

\subsubsection{Diseño}
	Esta pantalla aparece despu\'es de seleccionar la opci\'on "`Modificar expediente m\'edico"' de la \IUref{IUMedico}{Pantalla de Medico}\label{IUMedico}. Aqu\'i el medico puede modificar el expeidente m\'edico de algun paciente en caso de que sea necesario.

\IUfig[.7]{gui/IUModificaExp}{IUModificaExp}{Pantalla de Modificar Expediente M\'edico.}

\subsubsection{Salidas}

	Expediente M\'edico actualizado.

\subsubsection{Entradas}
Observaciones o datos modificados por parte del m\'edico.

\subsubsection{Comandos}
\begin{itemize}
		\item \IUbutton{Buscar}: El m\'edico seleccionar\'a la opcion de Buscar, ingresar\'a CURP y el sistema mostrar\'a el expediente resultante, que se modificar\'a.
		\item \IUbutton{Modificar}: El m\'edico seleccionar\'a la opcion de Modificarr y el sistema actualizar\'a el Expediente m\'edico.
		\item \IUbutton{Regresar}: Regresa al men\'u del m\'edico.
\end{itemize}

\subsubsection{Mensajes}
	\begin{Citemize}
		\item {\bf WARNDatosErr} - "`Dato ingresado incorrectamente."'.
		\item {\bf ERRExpInex} - "`El expediente m\'edico no existe en el sistema."'.
	\end{Citemize}