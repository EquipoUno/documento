% \IUref{IUAdmPS}{Administrar Planta de Selección}
% \IUref{IUModPS}{Modificar Planta de Selección}
% \IUref{IUEliPS}{Eliminar Planta de Selección}

% 


% Copie este bloque por cada caso de uso:
%-------------------------------------- COMIENZA descripción del caso de uso.

%\begin{UseCase}[archivo de imágen]{UCX}{Nombre del Caso de uso}{
	\begin{UseCase}{CU10}{Registrar Usuario}{
		El director de la clínica se encargará de registrar al personal nuevo dentro de la clínica (dependiente,cajero,médico y proveedor).
	}
		\UCitem{Versión}{1.1}
		\UCitem{Actor}{Directivo}
		\UCitem{Propósito}{Que el actor pueda registrar al personal nuevo y asignarlo a su respectiva área}
		\UCitem{Resumen}{
		Se ingresa el nombre del usuario y su contraseña,que usará cada uno del personal,el actor puede registrar al personal nuevo en la Base de Datos
		 }
		\UCitem{Entradas}{Nombre de usuario y contraseña}
		\UCitem{Salidas}{Pantalla de confrmación con datos escritos en ella}
		\UCitem{Precondiciones}{El personal por argegar debe tener ya un rol designado}
		\UCitem{Postcondiciones}{ La base de datos  guarda al nuevo personal}
		\UCitem{Errores}{El personal ya está registrado, las contraseñas ingresadas no coinciden.}
		\UCitem{Autor}{Ricardo Cuéllar Sánchez}
		\UCitem{Revisor}{Maldonado Ledo Diana Guadalupe}
		\UCitem{Status}{Terminado}
	\end{UseCase}

	\begin{UCtrayectoria}{Registro de Personal}
	\UCpaso[\UCactor] selecciona la opci\'on "Agregar Usuario"  de la \IUref{IURegUsuario}{Pantalla de Registro de Usuario} \label{CU10RegistroUsuario}.
		\UCpaso[\UCactor] Ingresa nombre de usuario y contraseña que desea resgistrar. 
		\UCpaso[\UCactor] Confirma la contraseña.
		\UCpaso El sistema válida si la contraseña escrita coincide con la primera.\Trayref{A}
		\UCpaso[\UCactor]Selecciona el rol que tomará el nuevo usuario.
		\UCpaso[\UCactor] Oprime el botón\IUbutton{Agregar}.\Trayref{B}
		\UCpaso El sistema manda el mensaje ``¿Seguro que quiere agregar a [Nombre] en [Área]?"\Trayref{C}
		\UCpaso[\UCactor] Presiona el botón \IUbutton{Aceptar}.
		\UCpaso El sistema muestra el mensaje ``Usuario: [Nombre] ha sido agregado éxitosamente a [Área]".
		\UCpaso El sistema guarda al nuevo usuario en la Base de Datos. 
		\UCpaso[\UCactor]Oprime el botón \IUbutton{Aceptar}.
	\end{UCtrayectoria}
		
		\begin{UCtrayectoriaA}{A}{Las contraseñas no coinciden.}
			\UCpaso El sistema muestra mensaje de error{\bf ERRContraseñaSinCoin-}
			\UCpaso[\UCactor]Oprime el botón \IUbutton{Aceptar}.
			\UCpaso[\UCactor] regresa al paso 2 de la trayectoria principal.  
			\UCpaso[] Termina la trayectoria.
		\end{UCtrayectoriaA}
		
		
	\begin{UCtrayectoriaA}{B}{El usuario ya existe en la Base de Datos}
			\UCpaso El sistema manda el mensaje {\bf ERRUsuarioExistente}
			\UCpaso[\UCactor]Oprime el botón \IUbutton{Aceptar}.
			\UCpaso El sistema limpia todos los datos. 
			\UCpaso[\UCactor] regresa al paso 1 de la trayectoria principal.  
			\UCpaso[] Termina la trayectoria.
		\end{UCtrayectoriaA}
		
		\begin{UCtrayectoriaA}{C}{}
			\UCpaso[\UCactor]Oprime el botón \IUbutton{Cancelar}
			\UCpaso El sistema limpia todos los datos de la pantalla. 
			\UCpaso[\UCactor] regresa al paso 1 de la trayectoria principal.  
			\UCpaso[] Termina la trayectoria.
		\end{UCtrayectoriaA}
		
%-------------------------------------- TERMINA descripción del caso de uso.