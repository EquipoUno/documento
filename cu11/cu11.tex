% \IUref{IUAdmPS}{Administrar Planta de Selección}
% \IUref{IUModPS}{Modificar Planta de Selección}
% \IUref{IUEliPS}{Eliminar Planta de Selección}

% 


% Copie este bloque por cada caso de uso:
%-------------------------------------- COMIENZA descripción del caso de uso.

%\begin{UseCase}[archivo de imágen]{UCX}{Nombre del Caso de uso}{
	\begin{UseCase}{CU13}{Crear Receta}{
		El m\'edico podr\'a crear un expediente en caso de que el paciente sea nuevo y no exista su historial en los registros de la cl\'inica, de esta manera tener un control sobre su historial.
	}
		\UCitem{Versión}{1.1}
		\UCitem{Actor}{M\'edico}
		\UCitem{Propósito}{Que el m\'edico se encargue de crear expedientes medicos de los pacientes que ingresaron por primera vez a la clinica.}
		\UCitem{Resumen}{
		El m\'edico seleccionar\'a la opci\'on para crear un expediente, solo si existe un paciente nuevo y de esta manera tener control sobre su historial cl\'inico, el paciente unicamente tendr\'a que proporcionar datos personales y familiares b\'asicos}
		\UCitem{Entradas}{Datos Personales.}
		\UCitem{Salidas}{Expediente terminado}
		\UCitem{Precondiciones}{No tener un expediente m\'edico en la cl\'inica, M\'edico deber\'a tener cuenta vigente.}
		\UCitem{Postcondiciones}{Una vez creado el expedicnete, unicamente el m\'edico encargado podr\'a hacer alguna modificaci\'on a dicho expediente.}
		\UCitem{Errores}{El paciente ya cuenta con un expediente m\'edico registrado en el sistema.}
		\UCitem{Autor}{Diana Guadalupe Maldonado Ledo}
		\UCitem{Revisor}{Gerardo Aramis Cabello Acosta}
		\UCitem{Status}{Espera de revision}
	\end{UseCase}

	\begin{UCtrayectoria}{Principal}
		\UCpaso[\UCactor] le indica al sistema que desea acceder a la acci\'on de crear expediente medico, seleccionando la opcion "`Crear Expediente"' del men\'u \IUref{IUMedico}{Pantalla de M\'edico}.
		\UCpaso El sistema muestra la \IUref{IUCreaExpediente}{Pantalla de Crear Expediente}\label{IUConsulta} .
		\UCpaso[\UCactor] El actor ingresa datos personales del paciente. 
		\UCpaso El sistema guarda en la BD los datos y valida los mismos. \Trayref{A}
		\UCpaso [\UCactor] selecciona la opción "`Terminar"'. 
		\UCpaso[] Termina el caso de uso.
	\end{UCtrayectoria}
		
		\begin{UCtrayectoriaA}{A}{Ya existe un expediente medico registrado}
			\UCpaso Muestra el mensaje {\bf ERRExpExiste} "El Expediente que desea crear, ya existe".
			\UCpaso[\UCactor] oprime el botón \IUbutton{Aceptar}
			\UCpaso[\UCactor] regresa al paso 1 de la trayectoria principal.  
			\UCpaso[] Termina la trayectoria.
		\end{UCtrayectoriaA}
		
		
		
%-------------------------------------- TERMINA descripción del caso de uso.