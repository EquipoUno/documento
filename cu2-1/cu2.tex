% \IUref{IUAdmPS}{Administrar Planta de Selección}
% \IUref{IUModPS}{Modificar Planta de Selección}
% \IUref{IUEliPS}{Eliminar Planta de Selección}

% 


% Copie este bloque por cada caso de uso:
%-------------------------------------- COMIENZA descripción del caso de uso.

%\begin{UseCase}[archivo de imágen]{UCX}{Nombre del Caso de uso}{
	\begin{UseCase}{CU2}{Efectuar compra}{
		Un dependiente se encarga de registrar la compra de medicamentos descritos en la receta, podrá imprimir un recibo que se pagará en Caja, y de esta manera, entregar al paciente los medicamentos que solicito.
	}
		\UCitem{Versión}{2.0}
		\UCitem{Actor}{Dependiente}
		\UCitem{Prop\'osito}{Que el actor pueda registrar una transacci\'on para ser terminada en caja, y otorgar un recibo de compra de medicamentos al paciente.}
		\UCitem{Resumen}{
		El actor pondr\'a en el sistema los medicamentos que est\'an indicados en la receta m\'edica, para posteriormente imprimir un recibo de pago con el resumen y un folio asignado por el sistema}
		\UCitem{Entradas}{ID de medicamento y cantidad de medicamento}
		\UCitem{Salidas}{Precio total de la compra y subtotales por medicamento.}
		\UCitem{Precondiciones}{El paciente debe tener una receta, ya sea de la cl\'inica misma, o de otra cl\'inica, el medicamento que se desea suministrar debe estar registrado en el sistema, y la cantidad de medicamentos a seleccionar no debe sobrepasar el stock.}
		\UCitem{Postcondiciones}{El paciente recibir\'a un recibo de compra de medicamentos.}
		\UCitem{Errores}{El medicamento no est\'a registrado en el sistema o se solicita una cantidad de medicamento mayor a la que se tiene en stock.}
		\UCitem{Autor}{Diana Guadalupe Maldonado Ledo}
		\UCitem{Revisor}{Gerardo Aramis Cabello Acosta}
		\UCitem{Status}{Revisado}
	\end{UseCase}

	\begin{UCtrayectoria}{Principal}
		\UCpaso[\UCactor] introduce el ID del medicamento y la cantidad de unidades a comprar en \IUref{IUCompra}{Pantalla de Compra}\Trayref{A}\Trayref{C}\Trayref{D}
		\UCpaso [\UCactor] le indica al sistema que desea comprar los medicamentos con las cantidades ingresadas seleccionando el bot\'on "`Comprar"'. \Trayref{E}
		\UCpaso El sistema muestra la \IUref{IUConfirmar}{Pantalla de Confirmaci\'on de compra}\label{IUConfirmar}.\Trayref{B}
		\UCpaso [\UCactor] le indica al sistema que desea confirmar la compra seleccionando la opci\'on "`Aceptar"'
		\UCpaso El sistema imprime un recibo que contiene la informaci\'on del paso 3 de la trayectoria principal y un n\'umero de transacci\'on generado al azar.
		\UCpaso[] Termina el caso de uso.
	\end{UCtrayectoria}
		
		\begin{UCtrayectoriaA}{A}{El sistema no encuentra el medicamento}
			\UCpaso Muestra el mensaje {\bf ERRMedInex-}``.
			\UCpaso[\UCactor] Oprime el botón \IUbutton{Aceptar}, regresa al paso $1$ de la trayectoria principal.  
		\end{UCtrayectoriaA}
		
	\begin{UCtrayectoriaA}{B}{Compra Cancelada.}
			\UCpaso[\UCactor]El actor cancela compra por falta de pago o por medicamento no disponible.
			\UCpaso El sistema muestra la \IUref{IUCompra}{Pantalla de Compra}\label{IUCompra}
			\UCpaso El [\UCactor] regresa al paso 1 de la trayectoria principal
		\end{UCtrayectoriaA}		
	\begin{UCtrayectoriaA}{C}{El [\UCactor] desea agregar m\'as edicamentos}
			
			\UCpaso[\UCactor] le indica al sistema que desea agregar otro medicamento, seleccionando el bot\'on "`Agregar"'
			\UCpaso El sistema agregar\'a un rengl\'on para poder ingresar un nuevo medicamento y una nueva cantidad del mismo.
				\UCpaso[\UCactor] regresa al paso 1 de la trayectoria principal.
		\end{UCtrayectoriaA}		
	\begin{UCtrayectoriaA}{D}{El [\UCactor] desea eliminar un medicamento ingresado}
			
			\UCpaso[\UCactor] le indica al sistema que desea eliminar un medicamento agregado, seleccionando el bot\'on "`Eliminar"'
			\UCpaso El sistema remueve el rengl\'on correspondiente al bot\'on al que el [\UCactor] seleccion\'o.
				\UCpaso[\UCactor] regresa al paso 1 de la trayectoria principal.
		\end{UCtrayectoriaA}
	\begin{UCtrayectoriaA}{E}{No hay suficiente stock de medicamentos para abastecer los medicamentos solicitados}
			
			\UCpaso El sistema muestra el mensaje {\bf ERRMedInsuf-}.
			\UCpaso El sistema remueve todos los renglones de medicamentos excepto el primero y pone los campos vac\'ios.
				\UCpaso[\UCactor] regresa al paso 1 de la trayectoria principal.
		\end{UCtrayectoriaA}
		
%-------------------------------------- TERMINA descripción del caso de uso.