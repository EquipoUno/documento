Introduction al capítulo

%--------------------------------------------------
\section{Glosario de términos}

Liste todos los termino del negocio.

\begin{description}
	\item \bfseries{Paciente de espera}\mdseries{.- Es aquel paciente que no tiene cita m\'edica y espera a que una cita no sea pagada para poder ser atendido en \'esta}
	\item \bfseries{Stock}\mdseries{.- Conjunto de mercanc\'ias o productos que se tienen almacenados en espera de su venta o comercializaci\'on}
	\item \bfseries{Expediente m\'edico}\mdseries{.- Espacio donde se almacena informaci\'on m\'edica de un paciente.}
	\item \bfseries{Medicamento controlado}\mdseries{.- Es aquel que s\'olo puede venderse teniendo receta m\'edica.}
	\item \bfseries{Laboratorio}\mdseries{.- Es aquel lugar cl\'inico d\'onde un medicamento determinado fue desarrollado.}
	\item \bfseries{Unidad}\mdseries{.- Es la forma en la que se distribuye un medicamento ($i.e.$ caja con $X$ tabletas).}
	
	\item \bfseries{Comprobante de Pago}\mdseries{.- Recibo en el que se demuestra que el paciente realiz\'o una trasaccion por pago en farmacia o en consulta }
	\item \bfseries{Sustancia Activa}\mdseries{.- Producto elaborado por la t\'ecnica farmac\'eutica del principio activo(sustancia medicinal) y de sus asociaciones o combinaciones destinadas a ser utilizadas en personas que tenga propiedades para prevenir, diagnosticar, tratar, aliviar o curar enfermedades.}
	\item \bfseries{ID Medicamento}\mdseries{.- Identificador \'unico para todos y cada uno de los medicamentos.}
	
\end{description}
