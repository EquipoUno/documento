

En \'este documento se exponen los problemas que la cl\'inica posee, las soluciones propuestas para los mismos, y los modelos que describen el funcionamiento de la cl\'inica, y el sistema destinado a resolver sus problemas. 
\'Este documento fue realizado por la organizaci\'on EquipoUNO para la cl\'inica de homeopat\'ia del IPN. Fue realizado el 5 de mayo de 2016 en las instalaciones de nuestra organizaci\'on.
EquipoUNO es una organizaci\'on compuesta por:
\begin{itemize}
\item Cabello Acosta Gerardo Aramis
\item Barrios Alvarado Daniel Alejandro
\item Maldonado Ledo Diana Guadalupe
\item Cuellar Sanchez Ricardo
\end{itemize}

%--------------------------------------------------
\section{Propósito}
El prop\'osito de \'este documento es explicar con detalle una serie de cosas como son:
\begin{itemize}
\item La forma en la que la cl\'inica opera actualmente, as\'i como qui\'enes est\'an involucrados en \'este proceso.
\item Definir todos los problemas que posee la cl\'inica y en qu\'e forma \'estos impactan.
\item Definir las propuestas que se tengan para resolver cada uno de los problemas definidos.
\item Definir el diseño que tendr\'a el sistema encargado de resolver las problem\'aticas definidas.
\item Definir los t\'erminos necesarios para un correcto entendimiento del documento.
\end{itemize}

%--------------------------------------------------
\section{Alcance}
El sistema tendr\'a una plataforma web que permitir\'a realizar citas m\'edicas via online, tendr\'a tres m\'odulos: Farmacia, Caja y Consultorio. 
El m\'odulo de la farmacia permitir\'a registrar compras de medicamento tanto en base a la receta expedida por la cl\'inica misma, como en base a otra. Tambi\'en permitir\'a consultar el stock de un medicamento en particular y realizar b\'usquedas de medicamento por medio de su nombre o susancia/s activa/s y expedir recibos de pago de medicamento.
El m\'odulo de la caja permitir\'a completar transacciones pendientes, ya sean de consultas o compra de medicamentos pendientes. \'Esto se har\'a por medio de un ID el cual se le ser\'a proporcionado al cliente por medio de un recibo de pago.
Finalmente el m\'odulo del consultorio le permitir\'a al m\'edico generar recetas m\'edicas y administrar expedientes m\'edicos (generarlos, modificarlos y almacenarlo) de forma electr\'onica.


%--------------------------------------------------
\section{Definiciones, acrónimos y abreviaturas}
\begin{itemize}
\item \bfseries{IU}\mdseries{.- Interfaz de usuario, es decir, lo que el usuario podr\'a ver en la pantalla del sistema}
\item \bfseries{ERR}\mdseries{.- Mensaje que indica un error as\'i como la causa que lo provoc\'o.}
\item \bfseries{OK}\mdseries{.- Mensaje que indica una operaci\'n exitosa}
\item \bfseries{WARN}\mdseries{.- Mensaje que indica una advertencia}
\item \bfseries{CONF}\mdseries{.- Mensaje que pide al usuario confirmar una acci\'on hecha previamente}
\item \bfseries{Bandera}\mdseries{.- Indicador que s\'olo puede ser verdadero o falso}
\item \bfseries{IMC}\mdseries{.- \'Indice de masa corporal. Es una medida de asociaci\'on entre la talla y el peso de una persona usada para evaluar el estado nutricional de la misma}
\item \bfseries{}\mdseries{.- }
\end{itemize}
%--------------------------------------------------
\section{Referencias}
Joshua M. Paiz, Elizabeth Angeli, Jodi Wagner, Elena Lawrick, Kristen Moore, Michael Anderson, Lars Soderlund, Allen Brizee, Russell Keck. (27/03/2015). Reference List: Electronic Sources (Web Publications). Purdue Online Writing Lab. Obtenido de: 
https://owl.english.purdue.edu/owl/resource/560/10/
%--------------------------------------------------

