% \IUref{IUAdmPS}{Administrar Planta de Selección}
% \IUref{IUModPS}{Modificar Planta de Selección}
% \IUref{IUEliPS}{Eliminar Planta de Selección}

% 


% Copie este bloque por cada caso de uso:
%-------------------------------------- COMIENZA descripción del caso de uso.

%\begin{UseCase}[archivo de imágen]{UCX}{Nombre del Caso de uso}{
	\begin{UseCase}{CU3}{Consultar Stock}{
		Un dependiente podrá consultar el stock de los medicamentos que actualmente existen en la farmacia y poder saber si una receta se podrá surtir en su totalidad.
	}
		\UCitem{Versión}{1.2}
		\UCitem{Actor}{Dependiente}
		\UCitem{Propósito}{Que el dependiente pueda conocer la cantidad de medicamentos que existen en la farmacia.}
		\UCitem{Resumen}{
		El dependiente buscar\'a un medicamente por su ID, y el sistema le mostrar\'a la cantidad de unidades que hay de ese medicamento, las cuales pueden ser cajas, botes, etc. \'Esto \'ultimo est\'a descrito en el atributo "`Descripci\'on"' del medicamento. }
		\UCitem{Entradas}{ID del medicamento.}
		\UCitem{Salidas}{Cantidad de unidades que hay en el stock, del medicamento buscado.}
		\UCitem{Precondiciones}{Tener registrado en el sistema el medicamento que se desea buscar.}
		\UCitem{Postcondiciones}{}
		\UCitem{Errores}{El medicamento no esta registrado en el sistema}
		\UCitem{Autor}{Diana Guadalupe Maldonado Ledo}
		\UCitem{Revisor}{Gerardo Aramis Cabello Acosta}
		\UCitem{Status}{Terminado}
	\end{UseCase}

	\begin{UCtrayectoria}{Principal}
		\UCpaso[\UCactor] le indica al sistema que desea acceder a la acci\'on de consulta, seleccionando la opcion "`Consultar Stock"' del men\'u superior \IUref{IUCompra}{Pantalla de Compra}.
		\UCpaso El sistema muestra la \IUref{IUConsulta}{Pantalla de Consulta}\label{IUConsulta} .
		\UCpaso[\UCactor] El actor ingresa el ID del medicamento a consultar. 
		\UCpaso El sistema muestra la cantidad de unidades que hay en stock, del medicamento buscado. \Trayref{A}
		\UCpaso [\UCactor] selecciona la opción "`Aceptar"'. 
		\UCpaso[] Termina el caso de uso.
	\end{UCtrayectoria}
		
		\begin{UCtrayectoriaA}{A}{El medicamento buscado no est\'a registrado en el sistema}
			\UCpaso Muestra el mensaje {\bf ERRMedInexis}.
			\UCpaso[\UCactor] oprime el botón \IUbutton{Aceptar}
			\UCpaso[\UCactor] regresa al paso 1 de la trayectoria principal.  
			\UCpaso[] Termina la trayectoria.
		\end{UCtrayectoriaA}
		
		
		
%-------------------------------------- TERMINA descripción del caso de uso.