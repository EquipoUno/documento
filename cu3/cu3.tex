% \IUref{IUAdmPS}{Administrar Planta de Selección}
% \IUref{IUModPS}{Modificar Planta de Selección}
% \IUref{IUEliPS}{Eliminar Planta de Selección}

% 


% Copie este bloque por cada caso de uso:
%-------------------------------------- COMIENZA descripción del caso de uso.

%\begin{UseCase}[archivo de imágen]{UCX}{Nombre del Caso de uso}{
	\begin{UseCase}{CU3}{Consultar Stock}{
		Un dependiente podrá consultar el stock de los medicamentos que actualmente existen en la farmacia y poder saber si una receta se podrá surtir en su totalidad.
	}
		\UCitem{Versión}{1.1}
		\UCitem{Actor}{Dependiente}
		\UCitem{Propósito}{Que el dependiente pueda conocer la cantidad de medicamentos que existen en la farmacia.}
		\UCitem{Resumen}{
		El dependiente seleccionara por sustacia activa o nombre del medicamento, y el sistema mostrará la cantidad de los medicamentos que se encuentren en el almacén de la farmacia. }
		\UCitem{Entradas}{Nombre o sustancia del medicamento.}
		\UCitem{Salidas}{Cantidad y nombre de los medicamentos que se encuentranen farmacia.}
		\UCitem{Precondiciones}{El dependiente deberá tener cuenta activa y encontrarse en la seccion de Consulta de Stock.}
		\UCitem{Postcondiciones}{Cantidad del medicamento buscado, mensajes de error.}
		\UCitem{Errores}{El medicamento no esta registrado en el sistema, cantidad nula de medicamentos.}
		\UCitem{Autor}{Diana Guadalupe Maldonado Ledo}
		\UCitem{Revisor}{}
		\UCitem{Status}{Espera de revisión}
	\end{UseCase}

	\begin{UCtrayectoria}{Principal}
		\UCpaso[\UCactor] El actor introduce la opcion "Consultar" del Menú superior \IUref{UIProv}{Pantalla de Compra}%\label{CU7IntroduceMedicamentos}.
		\UCpaso El sistema muestra la pantalla \IUref{UIProv}{Pantalla de Compra} .
		\UCpaso[\UCactor] El actor ingresa el Nombre o Sustancia Activa del medicamento a consultar. 
		\UCpaso El sistema muestra los medicamentos por la busqueda realizada, consultando el stock. \Trayref{A}
		\UCpaso [\UCactor] El actor selecciona la opción "Aceptar" 
		\UCpaso[] Termina el caso de uso.
	\end{UCtrayectoria}
		
		\begin{UCtrayectoriaA}{A}{Medicamento no existente}
			\UCpaso Muestra el Mensaje {\bf MSG1-}``No hay cantidad disponible en el stock del medicamento buscado.''.
			\UCpaso[\UCactor] Oprime el botón \IUbutton{Aceptar}, regresa al paso anterior [{\em CU3.4}].  
			\UCpaso[] Termina la trayectoria.
		\end{UCtrayectoriaA}
		
		
		
%-------------------------------------- TERMINA descripción del caso de uso.