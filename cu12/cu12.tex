% \IUref{IUAdmPS}{Administrar Planta de Selección}
% \IUref{IUModPS}{Modificar Planta de Selección}
% \IUref{IUEliPS}{Eliminar Planta de Selección}

% 


% Copie este bloque por cada caso de uso:
%-------------------------------------- COMIENZA descripción del caso de uso.

%\begin{UseCase}[archivo de imágen]{UCX}{Nombre del Caso de uso}{
	\begin{UseCase}{CU12}{Modificar Expediente}{
		El m\'edico podr\'a modificar un expediente en caso solo para usos medicos y que se requiera el seguimiento de un tratamiento.
	}
		\UCitem{Versión}{1.2}
		\UCitem{Actor}{M\'edico}
		\UCitem{Propósito}{Que el m\'edico se encargue de modificar expedientes medicos de los pacientes que necesitan de observaciones, seguimiento o tratamiento.}
		\UCitem{Resumen}{
		El m\'edico seleccionar\'a la opci\'on para modificar un expediente, solo si existe la necesidad de manetener un seguimiento, tratamiento o simplemente para agregar observaciones de un paciente.}
		\UCitem{Entradas}{Observaciones o Tratamiento}
		\UCitem{Salidas}{Expediente actualizado}
		\UCitem{Precondiciones}{M\'edico deber\'a tener cuenta vigente.}
		\UCitem{Postcondiciones}{El expediente ser\'a actualizado con la informaci\'on que el doctor modific\'o}
		\UCitem{Errores}{La cuenta del medico no este habilitada, caida del sistema, fallas electricas.}
		\UCitem{Autor}{Diana Guadalupe Maldonado Ledo}
		\UCitem{Revisor}{Ricardo Cuellar Sanchez}
		\UCitem{Status}{Terminado}
	\end{UseCase}

	\begin{UCtrayectoria}{Principal}
		\UCpaso[\UCactor] le indica al sistema que desea acceder a la acci\'on de modificar expediente medico, seleccionando la opcion "`Modificar Expediente"' del men\'u \IUref{IUMedico}{Pantalla de M\'edico}.
		\UCpaso El sistema muestra la \IUref{IUModificaExpediente}{Pantalla de Modificar Expediente}\label{IUConsulta} .
		\UCpaso[\UCactor] ingresa observaciones o tratamiento a seguir del paciente a quien atendi\'o en consulta. 
		\UCpaso El sistema guarda en la BD los datos agregados y valida los mismos. \Trayref{A}
		\UCpaso [\UCactor] selecciona la opción "`Guardar"'. 
		\UCpaso[] Termina el caso de uso.
	\end{UCtrayectoria}
		
		\begin{UCtrayectoriaA}{A}{Modificacion candelada}
			\UCpaso Muestra el mensaje {\bf ERRMdoficiacionExpedienteCancelada} "Usted ha cancelado la operacion".
			\UCpaso[\UCactor] oprime el botón \IUbutton{Aceptar}
			\UCpaso[\UCactor] regresa al paso 1 de la trayectoria principal.  
			\UCpaso[] Termina la trayectoria.
		\end{UCtrayectoriaA}
		
		
		
%-------------------------------------- TERMINA descripción del caso de uso.